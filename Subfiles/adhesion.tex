For the hydrogel to be effective, it must be able to stay in place once applied. This can prove to be a challenge for standard hydrogels, as daily movements can cause hydrogels with insufficient adhesiveness to migrate or fragmentation, rendering hydrogel application ineffectual. Furthermore, adhesive hydrogels help facilitate the deposition of a drug into targeted tissues rather than surrounding tissues by increasing direct surface contact and reducing the diffusion barrier \autocite{freedmanEnhancedTendonHealing2022}. Fortunately, the bioinspired JTA hydrogel shows strong adherence to tendons, integrating a chitosan adhesive surface that exerts an adhesive force of $>1000J/m^2$ \autocite{freedmanEnhancedTendonHealing2022}. Furthermore, while hydrogels must remain adhesive so their target tissue of interest, it must still allow free movement of adjacent healthy tissues to reduce interference with everyday function. A notable advantage of the JTA hydrogel is that only one side is adhesive, while the other side supports tissue and tendon gliding, hence its name “Janus” tough adhesive. The strong adhesiveness, as well as the hydrogel’s two-sidedness provides the hydrogel design with superior efficacy compared to traditional hydrogels.

\subsection{Mechanism of Adhesion}
The principal mechanism for the JTA hydrogel’s adhesion is bioinspired by the slug \textit{Arion subfuscus’s} defense mucus, which can strongly adhere to wet surfaces by using positively charged proteins coupled to a tough matrix \autocite{li_tough_2017}. Similarly, the JTA hydrogel consists of a dissipative tough matrix, which is coupled to amine-rich chitosan that can form positive charges upon the acceptance of an H+ ion \autocite{freedmanEnhancedTendonHealing2022}. 
To adhere to surfaces, the hydrogel bonds to substrates using electrostatic interactions, covalent bonding, and physical interpenetration \autocite{li_tough_2017}. The adhesion mechanism begins with the hydrogel’s chitosan polymers, which are absorbed to tendon surfaces by electrostatic interactions. This provides the primary amine groups of chitosan to bind covalently with carboxylic acids from both the hydrogel matrix and tendon surface. Finally, the chitosan can penetrate the tendon surface, forming physical entanglements in addition to chemical anchoring. Chitosan’s richness in amine’s provides it with its great adhesion energy of over $1000J/m^2$ \autocite{li_tough_2017}.

\subsection{Tendon Gliding}
While one surface of the hydrogel must be adhesive to a tissue of interest for effective use, the opposite side should be non-adhesive. This is because a common limitation of hydrogel healing is fibrotic scar formation, which is caused due to the increased friction between the healing tendon and surrounding tissue from adhesive hydrogels. Thus, any hydrogel used for treatment should support tendon gliding; that is, hydrogels should maintain low friction and low adhesiveness between other neighbouring tissues. Due to the mechanism of adhesion, where chitosan is loaded between the tendon surface and the alginate-polyacrylamide tough matrix, the non-adhesive tough matrix contacts neighbouring tissues. This allows for very low coefficients of friction when in contact with adjacent tissues, and mimics the amount of friction generated by other tissues \autocite{freedmanEnhancedTendonHealing2022}.
