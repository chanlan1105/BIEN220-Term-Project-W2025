Tendon injuries have high morbidities and often yield unsatisfactory treatment outcomes. This is due to the tendons' overreliance on extrinsic healing methods to regenerate damaged tissues post-injury. Extrinsic healing refers to the migration of inflammatory cells and fibroblasts to the injured tissue and is associated with the formation of scar tissue and adhesion between tissues. Intrinsic healing is governed by tenocytes and results in scarless tissue regeneration and improved patient outcomes. Due to the hypocellularity and hypovascularity of tendon tissues, intrinsic healing capacity in tendons is reduced and the quality of the regenerated tissue is structurally and mechanically inadequate. Tendon healing follows a triphasic process, which consists of the inflammatory phase, the proliferative phase and the remodelling phase. Improved healing outcomes following tendon injury would involve an enhanced intrinsic healing response in tendons throughout this healing process. Furthermore, an optimal course of treatment would mitigate the detriments associated with a dominant extrinsic healing response to injury \autocite{intro}.

Tissue engineering methods offer promising solutions in the development of improved tendon healing therapies. Engineered biomaterials aim to mimic the native biochemical and biophysical properties of the extracellular matrix for desirable outcomes in cell differentiation and proliferation. In conjunction with the appropriate growth factors and stem cells, these materials have been shown to enhance regular healing responses, provided the choice of material optimizes the mechanical and chemical support of tissue regeneration. Among these biomaterials, hydrogels demonstrate considerable promise in improving the regeneration of connective tissue \autocite{intro2}. Their composition provides a reliable platform for native cell proliferation and differentiation as is it analogous to that of the ECM. Furthermore, their structure is optimal for onsite delivery of stem cells and growth factors which bestows opportunities for clinicians to modulate the healing response toward more intrinsic-dependent responses post-operation \autocite{intro}. In this paper, we explore the limitations of hydrogel use and their potential solutions through a novel hydrogel design for improved postoperative tendon regeneration. Additionally, we discuss our methods of design and testing and present the model's own limitations.