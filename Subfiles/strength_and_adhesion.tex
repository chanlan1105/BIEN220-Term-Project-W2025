In addition to biocompatibility, the mechanical properties of a given hydrogel decide whether or not a given hydrogel is functional. In particular, hydrogels must have sufficient toughness and adhesive strength \autocite{Freedman.EnhancedTendonHealing}. In vivo, tendons must withstand a dynamic environment and bear strong forces \autocite{Chen.AdvancesApplicationHydrogel}; thus, a hydrogel must be able to withstand sufficient amounts of force without fracturing – that is, an ideal hydrogel should only deform plastically (Freedman et al.). Furthermore, to promote optimal healing, hydrogels should be placed and remain near the relevant tendon. However, force generated by movement can lead to hydrogel displacement; Hence, adhesion is required to ensure immobility (Freedman et al.). A variety of strategies can be implemented on a biochemical level to greatly increase the toughness and adhesiveness in order to create an adequate hydrogel for tendon repair.

This is still a test
High mechanical toughness of the JTA was achieved using a dual interpenetrating hydrogel network that combines the ability of alginate hydrogels to dissipate energy through dissociation of ionic bonds with a highly elastic covalently cross-linked acrylamide hydrogel that can distribute stresses throughout the network. \cite{FreedmanEnhancedTendonHealing}

The backbone structure of the tough SL was polyacrylic acid (PAA) networks, with long-chain glycidyl methacrylate-modified chondroitin sulfate (CSMA) as the cross-linking agent and encapsulated chondroitin sulfate (CS), which endowed the JHA with strong mechanical properties for effectively dissipating energy under deformation \cite{JuSurfaceEnzymePolymerization}

The interlocking structure and continuous features of the JHA were also confirmed from the Raman mapping image in Fig. 2i. This interlocked structure firmly bonded the two layers, and the RL could effectively transfer the strain to the SL through the interlocking structure, avoiding fracture and delamination, thereby achieving stable and robust adhesion to biological tissues. \cite{JuSurfaceEnzymePolymerization}

