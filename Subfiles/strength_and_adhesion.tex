In addition to biocompatibility, the mechanical properties of a given hydrogel decide whether or not a given hydrogel is functional. In particular, hydrogels must have sufficient toughness and adhesive strength (Freedman et al.). In vivo, tendons must withstand a dynamic environment and bear strong forces (Chen et al.); thus, a hydrogel must be able to withstand sufficient amounts of force without fracturing – that is, an ideal hydrogel should only deform plastically (Freedman et al.). Furthermore, to promote optimal healing, hydrogels should be placed and remain near the relevant tendon. However, force generated by movement can lead to hydrogel displacement; Hence, adhesion is required to ensure immobility (Freedman et al.). A variety of strategies can be implemented on a biochemical level to greatly increase the toughness and adhesiveness in order to create an adequate hydrogel for tendon repair.