\subsection{Janus Interfaces for Optimal Hydrogel Performance}
Hydrogels often integrate janus surfaces to exhibit a boundary-defined contrast in material properties. This boundary can be delimited microscopically or macroscopically and may be either pattern-based or cross-sectional (the material properties will change at the opposing faces of a cross-sectional plane). The Janus surface is a refinement of the Janus bead, developed in 1989, which consists of an amphiphilic glass bead with a cross-sectional wettability contrast \autocite{janus_interface}. The integration of varying wettabilities in a single material is biomimetic: it is inspired by both the carapace of the desert beetle \textit{Stenocara sp.} and its unique water retrieval mechanism as well as the surface of the lotus leaf and its contrasting wettabilities at the air and water interfaces \autocite{janus_interface}. Janus interface materials now integrate other boundary-contrasted material properties, such as contrasting chemical composition and mechanics. 

\subsection{Janus Tough Adhesives}
The Janus Tough Adhesive is a hydrogel that incorporates a Janus interface and is optimized for biomedical applications, particularly in tissue regeneration for tendon repair \autocite{jta_poc}. One face of the Janus interface allows for strong adhesion of the hydrogel to wet tissue surfaces and the other is designed to facilitate the gliding of surrounding tissues. The latter property is essential to prevent postoperative tissue adhesion, which is a frequent complication of invasive procedures, often resulting in reoperation. The unique properties of the JTA arise from the mixing of ionically crosslinked alginate -with calcium ions- and covalently crosslinked polyacrylamide \autocite{jta_poc}. Application of a chitosan-containing mixture on one side results in the highly adhesive character at the tendon interface of the hydrogel. JTAs are remarkable as they integrate either adhesive or gliding properties at its interfaces while maintaining uniformly high toughness. 

