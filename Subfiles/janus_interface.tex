\subsection{Janus Interfaces for Optimal Hydrogel Performance}
Hydrogels often integrate janus surfaces to exhibit a boundary-defined contrast in material properties. This boundary can be delimited microscopically or macroscopically and may be either pattern-based or cross-sectional (the material properties will change at the opposing faces of a cross-sectional plane). The Janus surface is a refinement of the Janus bead, developed in 1989, which consists of an amphiphilic glass bead with a cross-sectional wettability contrast \autocite{janus_interface}. The integration of varying wettabilities in a single material is biomimetic: it is inspired by both the carapace of the desert beetle \textit{Stenocara sp.} and its unique water retrieval mechanism as well as the surface of the lotus leaf and its contrasting wettabilities at the air and water interfaces \autocite{janus_interface}. Janus interface materials now integrate other boundary-contrasted material properties, such as contrasting chemical composition and mechanics. 

\subsection{Janus Tough Adhesives}
The Janus Tough Adhesive is a hydrogel that incorporates a Janus interface and is optimized for biomedical applications, particularly in tissue regeneration for tendon repair \autocite{freedmanEnhancedTendonHealing2022}. One face of the Janus interface allows for strong adhesion of the hydrogel to wet tissue surfaces and the other is designed to facilitate the gliding of surrounding tissues. The latter property is essential to prevent postoperative tissue adhesion, which is a frequent complication of invasive procedures, often resulting in reoperation. The unique properties of the JTA arise from the mixing of ionically crosslinked alginate -with calcium ions- and covalently crosslinked polyacrylamide \autocite{freedmanEnhancedTendonHealing2022}. Application of a chitosan-containing mixture on one side results in the highly adhesive character at the tendon interface of the hydrogel. JTAs are remarkable as they integrate either adhesive or gliding properties at its interfaces while maintaining uniformly high toughness. 

\subsection{Synthesis of JTAs}
Synthesizing the JTA requires the preparation of a 10 mL syringe containing 2\% sodium alginate and 12\% acrylamide. The syringe should contain high density and low density sodium alginate in a 1:1 ratio, such as MVG and VLVG sodium alginate. This syringe should be mixed with HBSS with $36 \mu L$ of 2\% N,N'-methylenebisacrylamide and $8 \mu L$ of TEMED: a cell culture medium solution, a polymer crosslinking agent and a polymerizing agent, respectively. This solution should be mixed with  $226 \mu L$ of 6.6\% ammonium persulfate and $191 \mu L$ of 0.75M calcium sulfate dihydrate, encapsulated within a glass mold, and set aside for 24 hours. Crosslinking will occur during this period. Once the 24 hours have elapsed, the hydrogel should be sliced in gel strips and the latter stored at 4 degrees celsius individually. A chitosan mixture is then to be applied and compressed on one of the faces of each strip with a surface density of $25 \mu l/cm{2}$. This mixture consists of 2\% chitosan with the following reagents mixed by vortexing: 1-ethyl-3-(3-dimethylaminopropyl) carbodiimide and sulfated N-hydroxy-succinimide \autocite{freedmanEnhancedTendonHealing2022}.

The synthesis of dynamic Schiff bases onto the chitosan structure requires bacterial cellulose, chopped by a high-speed homogenizer and added to a 0.10~M HCl solution. Adding sodium periodate to the mixture, it must then be allowed to react in the dark at 40~\textsuperscript{$\circ$}C for 10--12 hours.
This results in dialdehyde bacterial cellulose (DABC), which must be washed with 2\% ethylene glycol to remove any unreacted periodate. The DABC must then be mixed into a phosphate-buffered saline solution, designed to increase pH and simulate the liquid environment of the human body before it can be finally mixed with the chitosan solution \autocite{liAllnaturalInjectableHydrogel2020}.