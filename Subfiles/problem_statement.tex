Hydrogels are gaining significant traction in drug delivery and tissue enginering applications because of their structural resemblance to the native extracellular matrix. However, their limited mechanical properties make them incapable of sustaining high levels of mechanical stress.
Because of these mechanical weaknesses, hydrogels lose their function over time, reducing their efficacy in biomedical applications \autocite{rammalAdvancesBiomedicalApplications2021}. Nonetheless, a hydrogel-based cell scaffold has the potential to improve tendon repair and recovery when these weaknesses are addressed.

Advances in self-healing hydrogels would allow the retention of desired chemical and biological properties, while extending their lifespan and mechanical resilience. For instance, self-healing gelatin and acrylate $\beta$-cyclodextrin hydrogels were used for bone regeneration in lieu of traditional gelatin hydrogels, and exhibited improved mechanical properties and higher resistance to axial strain \autocite{rammalAdvancesBiomedicalApplications2021}.
The improved performance of self-healing hydrogels in bone regeneration could result in better outcomes when used for tendon repair.

In addition, drug release in hydrogels tends to occur in bursts, resulting in sharp fluctuations in drug concentration. A surge in concentration during release followed by a period of rapid depletion are undesirable, and the hydrogel solution should have tunable properties to ensure controlled drug delivery \autocite{freedmanEnhancedTendonHealing2022}.

Lastly, traditional hydrogels are difficult to fix in place once they are delivered by injection or surgical application. Particularly in movement-prone areas of the body such as joints, this leaves a possibility for physical displacement of the hydrogel scaffold. It therefore is necessary to develop an appropriate tissue adhesive for hydrogels to fulfill their objectives in tendon repair.