Hydrogels are 3D crosslinked polymer structures containing hydrohpilic functional groups, allowing them to absorb large quantities of water. Because of this, they are flexible and soft, and resemble many natural tissues \autocite{hoHydrogelsPropertiesApplications2022}.
Recent advances in hydrogel technology have led to the development of implantable and injectable hydrogels with potential applications in drug delivery. By adjusting polymer composition, key properties such as swelling-deswelling rate, stiffness, and degradability can be fine-tuned to meet specific use case requirements. As biomedical applications of hydrogels continue to expand, their use in tendon and ligament repair presents a promising opportunity.

Hydrogel cross-linked chains can be formed using natural, synthetic, or semi-synthetic polymers. Natural polymers such as cellulose, chitosan, and collagen are inherently biocompatible and bioactive, but come at the cost of weak stability and poor mechanical strength. Being derived from natural sources, these hydrogels are generally safe to use in clinical applications, but have shown to be allergens in rare cases \autocite{hoHydrogelsPropertiesApplications2022}.

On the other hand, synthetic hydrogels are made of man-made polymers like polyvinyl alcohol (PVA), polyethylene glycol (PVG), or polyacrylamide (PAAM). Few of these synthetic materials have been shown to be biocompatible, but they offer superior mechanical strength and stability.

To achieve both the biocompatibility offered by natural hydrogels, and the strength and mechanical properties offered by their synthetic counterparts, a common approach is to develop a hybrid, or semi-synthetic hydrogel chain. 