When subjected to physical or chemical stimuli, ``smart'' hydrogels have the ability to dynamically modify their physical properties, such as mechanical strength and swellability \autocite{hoHydrogelsPropertiesApplications2022}.
Self-healing hydrogels are a subset of smart hydrogels. These hydrogels are made of dynamic cross-linkages that can spontaneously re-form after having sustained mechanical damage. Self-healing properties allow the hydrogel to have superior mechanical toughness and a longer life span \autocite{deviv.k.SelfHealingHydrogelsPreparation2021}.

Self-healing hydrogels are unique in that the polymer chains are bound together by dynamic cross-linkages that can spontaneously self-repair. These cross-linkages can be the result of covalent or non-covalent interactions.
Covalent cross-linkages include imine and disulfide bonds. They are higher in energy and are therefore more difficult to break. Because of this, they are generally more stable and have superior mechanical strength.
Reversible non-covalent dynamic linkages arise from interactions such as hydrogen bonding, ionic interactions, and hydrophobic effects. These bonds are easily broken and reconstructed, which leads to a more flexible but mechanically weaker hydrogel \autocite{deviv.k.SelfHealingHydrogelsPreparation2021}.

\subsection{Self-Healing by Imine Bonds}

Imine bonds for self-healing hydrogels show promise in biomedical applications, especially in wound healing and in tissue engineering. Imines are chemical compounds containing a carbon atom doubly-bonded to a nitrogen. They are stabilised when the nitrogen atom contains an aromatic ring as a substituent, in which case it is known as a Schiff base \autocite{moldoveanuChapter8Pyrolysis2019}.
The C=N imine bond is formed when an aldehyde reacts with an amine, forming the Schiff base. This reaction is reversible, meaning that the imine bond can break and reform spontaneously given the appropriate conditions.