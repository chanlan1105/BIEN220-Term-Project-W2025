In consequence to slow tendon healing, as well as the limitations of tendon repair with traditional hydrogels, the primary objective is thus to design a hydrogel that utilizes biomimetic elements to more rapidly repair tendons, while still allowing for general movement. To achieve this, three specific aims must be addressed for a sufficient hydrogel. 
\begin{enumerate}
\item The hydrogel must have adequate mechanical and adhesive properties to support movement.
\item The hydrogel should have self-healing properties for increased longevity, while still being able to degrade adequately over time.
\item The hydrogel must be able to deliver healing and growth factors in a sustainable manner, and target tendon tissue specifically.
\end{enumerate}
Meeting these aims would satisfy the requirements for a hydrogel-based tendon repair method, which can be validated through in-vivo testing, discussed later.
