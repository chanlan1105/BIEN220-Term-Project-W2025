The proposed Janus Tough Adhesive hydrogel drug delivery system has a fracture energy of 9000~J/m\textsuperscript{2}, far higher than conventional hydrogels (100~J/m\textsuperscript{2}) and natural cartilage (1000~J/m\textsuperscript{2}), and on a similar scale a natural rubbers (10,000~J/m\textsuperscript{2}).
It uses imine bonding to dynamically re-form its cross-linkages should they fracture, increasing its lifespan.
The biomimetic adhesive has also shown adhesive strength exceeding 1000~J/m\textsuperscript{2}, and the reverse side mimics the friction generated by other biological tissues, allowing for natural tendon gliding.
Furthermore, the JTA has been shown to increase drug-loading capabilities 5000-fold when compared to current capacities (0.01--1~mg/mL). Drugs are released as the polymer network breaks down, which can be tuned to occur synchronously with healing.
The JTA's monomers are biocompatible and are metabolised by the liver or excreted by the body's urinary system.
As a result of its mechanical properties, drug-loading capabilities, and tuneable degradation rate, the proposed JTA drug delivery system is an ideal candidate for tendon repair applications. Altogether, it reduces the need for invasive treatments and accelerates recovery, leading to more effective, patient-friendly therapies for musculoskeletal injuries.