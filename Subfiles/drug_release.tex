\subsection{Standard of Care for Drug-Based Therapies}
Hydrogel-based therapies for connective tissue injuries enable improved mechanisms of drug delivery. Novel mechanisms of drug delivery seek to address many of the issues plaguing the effective treatment of disease with drugs. For one, the modest successes of many drug delivery mechanisms are diminished by the drug's bioavailability via its route of administration \autocite{nsaids}. Furthermore, traditional routes of drug administration often require frequent intervals of high drug dosages to induce noticeable effects. These drug administration modalities often diminish the drug-s overall efficacy by reducing patient compliance and increasing overall toxicity to the body \autocite{nsaids}. Drug performance can be improved via localized sustained release, i.e., subjecting the diseased region to uniform and lower drug concentrations locally throughout the course of treatment.

\subsection{Optimization of Hydrogel Composition for Sustained Drug Release}
The tunable nature of the hydrogel makes it a promising candidate to actuate localized and sustained drug release. Hydrogels are hierarchical in nature: macroscopic properties define the route of administration, while the microscopic scale governs convective drug transport and other physical interactions \autocite{mooney_drug}. The mesh scale -several hundred nanometers- governs diffusion of drugs within the polymer network and the molecular scale defines binding interactions between the drugs and the polymers. Hydrogel features at some given length scale can be modified while preserving the hydrogel's general character, i.e., the hydrogel can be modified to optimally suit a medication regimen while retaining its macroscopic mechanical and physical properties \autocite{mooney_drug}.  Relatedly, the high water content within hydrogels entails minimal denaturation of the drugs that are stored within the mesh space. Likewise, the extremely small mesh size resulting from polymer interactions prevents the inward diffusion of enzymes into the matrix and precludes premature degradation of the drugs. 

\subsection{JTAs for Sustained Drug Release}
The mechanical and physical properties of the JTA make it an ideal candidate for the improvement of current drug depot mechanisms. While most depots cannot adhere to the targeted tissue, the adhesive nature of the chitosan coating on the JTA restrains movement and prevents migration on the tissue. Likewise, the hydrogel resists mechanical disintegration due to its toughness and can thus avert burst release of the drug. The miniscule mesh size of the JTA impairs diffusion-based drug release, resulting in the almost exclusive release of a drug by dissolution, which suggests an effective method for delivering hydrophobic drugs to the diseased tissue \autocite{jta_poc}. Using triamcinolone acetonide as a model drug, a corticosteroid, it was found that the hydrogel could sustain stable drug release over a 10-day interval under perfect sink conditions. Additionally, while current standard of care suffers from poor loading capacities (0.01-1mg/ml), the JTA can support loading of drug concentrations up to 5000 times this amount, emphasizing the benefit of its use for long-term therapeutic goals \autocite{jta_poc}. Targeted drug therapies present a unique opportunity to tailor medication regimens to preserve the beneficial aspects of some drugs while mitigating their undesirable effects at a local scale.

