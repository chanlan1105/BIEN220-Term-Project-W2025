As the goal of a hydrogel scaffold is to promote natural healing, it must degrade over time to
allow the naturally-formed tissue to take over, yet its lifespan must be long enough to
continuously support the afflicted area throughout the healing process. Tendon and ligament
repair is a lengthy process that can take several months to over a year for full healing \autocite{RN1}. Traditional polymer-based hydrogels degrade too quickly to be effective in long healing
processes. Depending on the specific chemical composition, these hydrogels can degrade
within the body in less than 6 months due to hydrolysis, oxidization, and enzymatic activity.
Therefore, the use of a composite hydrogel becomes absolutely necessary for use in
tendon/ligament repair.

Composite hydrogels incorporate blends of natural and synthetic materials and linkage
structures such as double networks and interpenetrating networks. These hydrogels not only
last longer than natural or synthetic polymer-based hydrogels, but are also tunable, as their
lifespan can be modified by adjusting the cross-linking type and density. This allows for
improved customizability while maintaining permeability and resisting premature degradation. In
addition, the hydrogel structure can be engineered to respond to environmental stimuli such as
pH and enzyme presence, allowing them to be dynamic and responsive to the state of the
healing tissue, gradually breaking down in sync with tissue regeneration \autocite{RN3}.

\subsection{Effects of Crosslinking on Hydrogel Degradation
}

As mentioned above, composite hydrogels are highly customizable. Their lifespan may be
customized to better fit the timeline of the specific injury being treated. One such technique is
modifying the crosslinking density between hydrogel layers. A study by \autocite{RN4} found that an
increased concentration of glutaraldehyde crosslinking correlates with higher mechanical
strength and lower degradation rates. Greater crosslinking forms a denser polymer network,
making the hydrogel more resistant to breakdown from oxidation, hydrolysis, and enzymatic
activity (\citeyear{RN4}).

\subsection{Methods of Testing Hydrogel Efficacy}

The efficacy of a hydrogel depends on several factors. Absorption, or swelling, tests,
approximated by placing the scaffold in a liquid medium and measuring the percentage increase
in weight. This serves as a measure of permeability, simulating the scaffold's ability to allow cells
and nutrients into the healing area. Stress-strain tests are used to approximate mechanical
strength, ensuring that the hydrogel will not collapse or overextend under mechanical stress.
Adhesion is measured via lap shear or peel tests \autocite{RN5}, which provide a numerical
measure for the ability of the hydrogel to resist movement after its placement. Degradation rate
can be measured in a couple ways, by measuring the overall weight of the scaffold at 2 or more
points in time and representing degradation as a percentage loss in mass or by measuring the
percentage change in the diameter of individual fibrils that make up the matrix. These tests are
effective at measuring individual properties of the matrix ex vivo and predicting the success of
the hydrogel before it is used for treatment. However, an overall test can also be conductive
post-treatment, simply by measuring the strength of the tendon or ligament relative to its original
strength.