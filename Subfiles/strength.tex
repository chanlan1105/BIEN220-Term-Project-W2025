The mechanical properties of a given hydrogel decide whether or not a given hydrogel is functional. In particular, hydrogels must have sufficient toughness and adhesive strength \autocite{freedmanEnhancedTendonHealing2022}. In vivo, tendons must withstand a dynamic environment and bear strong forces \autocite{ChenAdvancesApplicationHydrogel}; thus, a hydrogel must be able to withstand sufficient amounts of force without fracturing – that is, an ideal hydrogel should only deform plastically \autocite{freedmanEnhancedTendonHealing2022}. Furthermore, to promote optimal healing, hydrogels should be placed and remain near the relevant tendon. However, force generated by movement can lead to hydrogel displacement; Hence, adhesion is required to ensure immobility \autocite{freedmanEnhancedTendonHealing2022}. A variety of strategies can be implemented on a biochemical level to greatly increase the toughness and adhesiveness in order to create an adequate hydrogel for tendon repair.

\subsection{Mechanisms for Toughness in Hydrogels}
In order to design a tough hydrogel, an analysis by Zhao (2014) found that tough hydrogels generally follow two principles; a tough hydrogel should dissipate significant amounts of mechanical energy upon crack propagation, and the original configuration of the hydrogel should be retained, even after large deformations. Simply put, tough hydrogels should have mechanisms to dissipate energy, and mechanisms to maintain elasticity. 

\subsection{Fracture of polymer chains}
A widely used method for energy dissipation is the fracturing of polymer chains \autocite{zhao_multi-scale_2014}. As a polymer chain is fractured, the mechanical energy stored within the chain is dissipated. Thus, to promote polymer fracturing, several polymer chains with short lengths are incorporated into the hydrogel. During a deformative event, these short chains can be ruptured to dissipate energy. Highly crosslinked polymers, such as polyacrylamide double network hydrogels, are notably effective in dissipating energy using this method \autocite{zhao_multi-scale_2014}. However, it should be noted that hydrogels that rely on chain-fracturing to dissipate energy are susceptible to fatigue over multiple deformations \autocite{zhao_multi-scale_2014}, highlighting the importance of a self repairing method, which will be discussed later.

\subsection{Reversible Crosslinking of Polymer Chains}
Another mechanism to dissipate energy is to decrosslink polymer chain networks. Physical crosslinkers found in polymer networks are usually weaker than covalent crosslinking, allowing for energy to be more easily dissipated through decrosslinking. In addition, the mechanism of breaking physical crosslinkers has the advantage of being recoverable, meaning that this mechanism of energy dissipation can maintain stress-strain hysteresis over cyclic loading, or this mechanism allows for anti-fatigue hydrogels. However, recovered cross linkers are often not found in their original locations post deformation – Zhao observed plastic deformations in ionically crosslinked alginate hydrogels – thus, a mechanism to maintain elasticity is critical.

\subsection{Interpenetration of long-chain networks}
According to \cite{rubinstein_networks_2003}, longer polymers allow for an increase in elasticity. Thus, long polymer chains can be interlaced with short-chain networks to form elastic interpenetrating polymer networks. Common polymer candidates for long-chain network elasticity include polyacrylamide, among other polymers.

\subsection{Hybrid crosslinking of physical and chemical crosslinkers}
As mentioned earlier, the recovery of physical crosslinkers often leads to irreversible deformations. Thus, chemical crosslinkers can be used to maintain the elasticity, while physical cross linkers can be used to dissipate energy. Resultant hybrid polymer networks are effective in maintaining elasticity and dissipating energy, and are thus used for many tough hydrogels. Examples of polymer networks that are used for hybrid cross linking include alginate, chitosan, and polyacrylamide \autocite{zhao_multi-scale_2014}.

\subsection{Application to the JTA}
The proposed JTA hydrogel uses an alginate-polyacrylamide polymer network to ensure hydrogel toughness, which uses all of the above mechanisms in order to retain superior mechanical toughness. Due to its use of the above mechanisms, the proposed hydrogel has a fracture energy of 9000 $J/m^2$, which is far superior to typical hydrogels, which typically have fracture energies around $10J/m^2$ . In comparison to human tissues, the proposed JTA hydrogel is stronger than cartilage, which has a fracture energy of $1000 J/m^2$, and slightly weaker than natural rubbers $(10,000 J/m^2)$ \autocite{sun_highly_2012}. Furthermore, in vivo load testing by Freedman et al. (2022) demonstrated that the alginate-polyacrylamide polymer hydrogel maintained its mechanical integrity. Therefore, the tough matrix in our proposed hydrogel demonstrates sufficient mechanical toughness for tendon repair.
